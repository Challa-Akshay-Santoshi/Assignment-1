\documentclass[journal,12pt,twocolumn]{IEEEtran}

\usepackage{setspace}
\usepackage{gensymb}
\singlespacing
\usepackage[cmex10]{amsmath}

\usepackage{amsthm}

\usepackage{mathrsfs}
\usepackage{txfonts}
\usepackage{stfloats}
\usepackage{bm}
\usepackage{cite}
\usepackage{cases}
\usepackage{subfig}

\usepackage{longtable}
\usepackage{multirow}

\usepackage{enumitem}
\usepackage{mathtools}
\usepackage{steinmetz}
\usepackage{tikz}
\usepackage{circuitikz}
\usepackage{verbatim}
\usepackage{tfrupee}
\usepackage[breaklinks=true]{hyperref}
\usepackage{graphicx}
\usepackage{tkz-euclide}

\usetikzlibrary{calc,math}
\usepackage{listings}
    \usepackage{color}                                            %%
    \usepackage{array}                                            %%
    \usepackage{longtable}                                        %%
    \usepackage{calc}                                             %%
    \usepackage{multirow}                                         %%
    \usepackage{hhline}                                           %%
    \usepackage{ifthen}                                           %%
    \usepackage{lscape}     
\usepackage{multicol}
\usepackage{chngcntr}

\DeclareMathOperator*{\Res}{Res}

\renewcommand\thesection{\arabic{section}}
\renewcommand\thesubsection{\thesection.\arabic{subsection}}
\renewcommand\thesubsubsection{\thesubsection.\arabic{subsubsection}}

\renewcommand\thesectiondis{\arabic{section}}
\renewcommand\thesubsectiondis{\thesectiondis.\arabic{subsection}}
\renewcommand\thesubsubsectiondis{\thesubsectiondis.\arabic{subsubsection}}


\hyphenation{op-tical net-works semi-conduc-tor}
\def\inputGnumericTable{}                                 %%

\lstset{
%language=C,
frame=single, 
breaklines=true,
columns=fullflexible
}
\begin{document}

\newcommand{\BEQA}{\begin{eqnarray}}
\newcommand{\EEQA}{\end{eqnarray}}
\newcommand{\define}{\stackrel{\triangle}{=}}
\bibliographystyle{IEEEtran}
\raggedbottom
\setlength{\parindent}{0pt}
\providecommand{\mbf}{\mathbf}
\providecommand{\pr}[1]{\ensuremath{\Pr\left(#1\right)}}
\providecommand{\qfunc}[1]{\ensuremath{Q\left(#1\right)}}
\providecommand{\sbrak}[1]{\ensuremath{{}\left[#1\right]}}
\providecommand{\lsbrak}[1]{\ensuremath{{}\left[#1\right.}}
\providecommand{\rsbrak}[1]{\ensuremath{{}\left.#1\right]}}
\providecommand{\brak}[1]{\ensuremath{\left(#1\right)}}
\providecommand{\lbrak}[1]{\ensuremath{\left(#1\right.}}
\providecommand{\rbrak}[1]{\ensuremath{\left.#1\right)}}
\providecommand{\cbrak}[1]{\ensuremath{\left\{#1\right\}}}
\providecommand{\lcbrak}[1]{\ensuremath{\left\{#1\right.}}
\providecommand{\rcbrak}[1]{\ensuremath{\left.#1\right\}}}
\theoremstyle{remark}
\newtheorem{rem}{Remark}
\newcommand{\sgn}{\mathop{\mathrm{sgn}}}
\newcommand{\comb}[2]{{}^{#1}\mathrm{C}_{#2}}
\providecommand{\abs}[1]{\vert#1\vert}
\providecommand{\res}[1]{\Res\displaylimits_{#1}} 
\providecommand{\norm}[1]{\lVert#1\rVert}
%\providecommand{\norm}[1]{\lVert#1\rVert}
\providecommand{\mtx}[1]{\mathbf{#1}}
\providecommand{\mean}[1]{E[ #1 ]}
\providecommand{\fourier}{\overset{\mathcal{F}}{ \rightleftharpoons}}
%\providecommand{\hilbert}{\overset{\mathcal{H}}{ \rightleftharpoons}}
\providecommand{\system}{\overset{\mathcal{H}}{ \longleftrightarrow}}
	%\newcommand{\solution}[2]{\textbf{Solution:}{#1}}
\newcommand{\solution}{\noindent \textbf{Solution: }}
\newcommand{\cosec}{\,\text{cosec}\,}
\providecommand{\dec}[2]{\ensuremath{\overset{#1}{\underset{#2}{\gtrless}}}}

\newcommand{\myvec}[1]{\ensuremath{\begin{pmatrix}#1\end{pmatrix}}}
\newcommand{\mydet}[1]{\ensuremath{\begin{vmatrix}#1\end{vmatrix}}}
\numberwithin{equation}{subsection}
\makeatletter
\@addtoreset{figure}{problem}
\makeatother
\let\StandardTheFigure\thefigure
\let\vec\mathbf
\renewcommand{\thefigure}{\theproblem}
\def\putbox#1#2#3{\makebox[0in][l]{\makebox[#1][l]{}\raisebox{\baselineskip}[0in][0in]{\raisebox{#2}[0in][0in]{#3}}}}
     \def\rightbox#1{\makebox[0in][r]{#1}}
     \def\centbox#1{\makebox[0in]{#1}}
     \def\topbox#1{\raisebox{-\baselineskip}[0in][0in]{#1}}
     \def\midbox#1{\raisebox{-0.5\baselineskip}[0in][0in]{#1}}
\vspace{3cm}
\title{Assignment 1}
\author{Challa Akshay Santoshi-CS21BTECH11012}
\maketitle
\newpage
\bigskip
\renewcommand{\thefigure}{\theenumi}
\renewcommand{\thetable}{\theenumi}
\begin{center}
  \textbf{\underline{ICSE 10 2018}}\\
\end{center}
\begin{center}
  \textbf{Question: 3 (c)}  
\end{center}
Using a graph paper draw a histogram for the given distribution showing the number of runs scored by 50 batsmen. Estimate the mode of the data.\\
\begin{table}[h!]
\centering
    \begin{tabular}{|p{0.30\linewidth}|p{0.25\linewidth}|}
    \hline
    \textbf{Runs scored} & \textbf{No. of batsmen}\\[0.5ex]
    \hline
    3000-4000 & 4\\
    \hline
    4000-5000 &18\\
    \hline
    5000-6000 & 9\\
    \hline
    6000-7000 & 6\\
    \hline
    7000-8000 & 7\\
    \hline
    8000-9000 & 2\\
    \hline
    9000-10000 & 4\\
    \hline
    \end{tabular}
\end{table}
\begin{center}
Table 1.1\\
\end{center}
\begin{center}
  \textbf{Solution:}  
\end{center}
The Histogram for the data given in Table 1.1 is plotted as shown in Fig: 1.1. \\
The approach for calculating mode is outlined in Fig: 1.2. The interval corresponding to the maximum number of batsmen is the mode class. The intersection of the lines PQ and RS as shown in Fig 1.2 (Point M) is the mode point. The required mode is the x-coordinate of the Mode point.\\
\begin{align}
	\vec{P} &= \myvec{5000 \\ 18},
	\vec{Q} = \myvec{4000 \\ 4},
	\\
	\vec{R} &= \myvec{4000 \\ 18},
	\vec{S} = \myvec{5000 \\ 9}
\end{align}

\begin{figure}[ht]
    \centering
    \includegraphics[width=\columnwidth]{fig 11.png}
    \caption{Histogram}
    \label{Fig 1.1}
\end{figure}
\begin{figure}[ht]
    \centering
    \includegraphics[width=\columnwidth]{fig 22.png}
    \caption{Histogram showing the mode}
    \label{}
\end{figure}
Equations of lines are as follows:\\
\begin{align}
    \vec{PQ} &\equiv 14x - 1000y =  52000 \\ \vec{RS} &\equiv 9x + 1000y =  54000 
\end{align}
Point of intersection of both these lines can be found using the vector approach as follows,\\
\begin{align}
\vec{X} &= \myvec{x \\ y}
\end{align}
\begin{align}
\vec{PQ} &\equiv \myvec{14 \\ -1000}\vec{X} = 52000\\
\vec{RS} &\equiv \myvec{9 \\ 1000}\vec{X} = 54000
\end{align}
\begin{align}
\myvec{14 & -1000 \\
		9 & 1000}\vec{X} &= \myvec{52000 \\ 54000}
\end{align}	
Augmented matrix for the matrix equations given above can be written as,\\
\begin{align}
	\myvec{ 14 & -1000 & \vrule & 52000 \\
		9 & 1000 & \vrule & 54000} \\
	\xleftrightarrow[]{R_1 \leftarrow R_1+R_2}
		\myvec{ 23 & 0 & \vrule & 106000 \\
			9 & 1000 & \vrule & 54000} \\
	\xleftrightarrow[]{R_2 \leftarrow 23R_2-9R_1}
		\myvec{ 23 & 0 & \vrule & 106000 \\
			0 & 23000 &\vrule & 288000} \\
	\xleftrightarrow[]{R_2 \leftarrow R_2/23000}
		\myvec{ 23 & 0 & \vrule & 106000 \\
			0 & 1 & \vrule & 12.521} \\
	\xleftrightarrow[]{ R_1 \leftarrow R_1/23}
		\myvec{ 1 & 0 & \vrule & 4608.695 \\
			0 & 1 & \vrule & 12.521} \\
	\implies \vec{X} = \myvec{4608.695 \\ 12.521}
\end{align}
Mode point is the point of intersection of lines PQ and RS.\\\\
Therefore mode point is\\
\begin{align}
	\vec{M} &= \myvec{4068.695 \\ 12.521}
\end{align}\\
Therefore, the mode is 4068.695.\\

\end{document}